\documentclass[12pt,a4paper]{article}
\usepackage[utf8]{inputenc}
\usepackage[left=1in,right=1in,top=1in,bottom=1in]{geometry}
\usepackage{graphicx}
\usepackage{hyperref}
\usepackage{array}
\usepackage{longtable}
\usepackage{float}
\usepackage{fancyhdr}
\usepackage{titlesec}

\pagestyle{fancy}
\fancyhf{}
\rhead{Full Stack Frameworks}
\lhead{23CSE461}
\cfoot{\thepage}

\begin{document}

\begin{titlepage}
    \centering
    \vspace*{2cm}
    
    {\Huge\bfseries FULL STACK FRAMEWORKS\\}
    \vspace{0.5cm}
    {\Large 23CSE461\\}
    \vspace{2cm}
    
    {\Large\bfseries LAB MANUAL\\}
    \vspace{3cm}
    
    \begin{tabular}{ll}
        \textbf{Roll No:} & 9059Rohith \\[0.3cm]
        \textbf{Name:} & Rohith \\[0.3cm]
        \textbf{Github URL:} & \href{https://github.com/9059Rohith/full_stack_assignment}{github.com/9059Rohith/full\_stack\_assignment} \\
    \end{tabular}
    
    \vfill
    
    {\large Department of Computer Science and Engineering\\}
    {\large \today\\}
\end{titlepage}

\newpage
\tableofcontents
\newpage

\section{Course Information}

\subsection{23CSE461 - FULL STACK FRAMEWORKS}

\subsubsection{Unit I - React JS}
Creating and using components, bindings, props, states, events, Working with components, Conditional rendering, Building forms, Getting data from RESTful APIs, Routing, CRUD with Firebase, Redux, React and Redux, Function vs. class based components, Hooks.

\subsubsection{Unit II - Express JS}
Node JS -- Basics, setup, console, command utilities, modules, events, Express JS -- Routing, HTTP methods, CSS, Bootstrap, JavaScript, React, Redux, Node, Express, URL building, Templates, Static files, Form data, Database, Cookies, Sessions, Authentication, RESTful APIs, Scaffolding, Error handling, Debugging.

\subsubsection{Unit III - Mongo DB}
Mongo DB ecosystem, Importing and Exporting data, Mongo query language, Updating documents, Aggregation framework, System and user generated variables, Schema validation, Data modelling, Indexing, Performance.

\subsection{Objectives and Outcomes}

\subsubsection{Course Objectives}
\begin{itemize}
    \item Web development has become easier with the introduction of frameworks.
    \item It has also paved the way for full stack web development.
    \item Full-stack developers use frameworks to develop, optimize and maintain websites and other web applications.
    \item This course covers some of the important full stack frameworks.
\end{itemize}

\subsubsection{Course Outcomes}
\begin{itemize}
    \item \textbf{CO1:} Learn how to develop single page applications (SPAs) efficiently using front-end framework.
    \item \textbf{CO2:} Learn to use backend frameworks to develop web and mobile applications robustly.
    \item \textbf{CO3:} Learn to build highly available and scalable internet applications using document databases.
    \item \textbf{CO4:} Design and develop full stack web projects using front-end, back-end and database frameworks.
\end{itemize}

\subsection{Reference Documentation URLs}

\begin{table}[H]
\centering
\begin{tabular}{|l|l|}
\hline
\textbf{Technology} & \textbf{Documentation URL} \\ \hline
ReactJS & \href{https://react.dev/}{https://react.dev/} \\ \hline
NodeJS & \href{https://nodejs.org/docs/}{https://nodejs.org/docs/} \\ \hline
MongoDB & \href{https://docs.mongodb.com/}{https://docs.mongodb.com/} \\ \hline
ExpressJS & \href{https://expressjs.com/}{https://expressjs.com/} \\ \hline
JavaScript & \href{https://developer.mozilla.org/en-US/docs/Web/JavaScript}{MDN JavaScript} \\ \hline
HTML & \href{https://developer.mozilla.org/en-US/docs/Web/HTML}{MDN HTML} \\ \hline
Responsive HTML & \href{https://web.dev/responsive-web-design-basics/}{Web.dev Responsive} \\ \hline
CSS & \href{https://developer.mozilla.org/en-US/docs/Web/CSS}{MDN CSS} \\ \hline
\end{tabular}
\end{table}

\newpage

\section{List of Exercises}

\begin{longtable}{|c|c|p{8cm}|c|}
\hline
\textbf{Sl.No} & \textbf{Ex.No.} & \textbf{Title of the Experiments} & \textbf{Page No.} \\ \hline
\endfirsthead

\hline
\textbf{Sl.No} & \textbf{Ex.No.} & \textbf{Title of the Experiments} & \textbf{Page No.} \\ \hline
\endhead

1 & 1 & Factorial, Fibonacci Series, and Prime Number Checker & \pageref{ex1} \\ \hline
2 & 2 & Sum of Digits Calculator & \pageref{ex2} \\ \hline
3 & 3 & Employee Tax Calculator (Class \& Function Components) & \pageref{ex3} \\ \hline
4 & 4 & Calculator Program & \pageref{ex4} \\ \hline
5 & 5 & Calculator with Math Game for Kids & \pageref{ex5} \\ \hline
\end{longtable}

\newpage

% ============ EXERCISE 1 ============
\section{EXERCISE NO: 1}
\label{ex1}

\subsection{AIM}
To implement a ReactJS application that performs the following operations:
\begin{itemize}
    \item Calculate factorial of a number
    \item Generate Fibonacci series up to N terms
    \item Check whether a given number is prime or not
\end{itemize}

\subsection{GitHub Repository}
\href{https://github.com/9059Rohith/full_stack_assignment/tree/main/ass_01/ass_01}{https://github.com/9059Rohith/full\_stack\_assignment/tree/main/ass\_01/ass\_01}

\subsection{Live Deployment}
\href{https://shimmering-brioche-55d13f.netlify.app/}{https://shimmering-brioche-55d13f.netlify.app/}

\subsection{LIST OF FILE NAMES WITH ITS PURPOSE}

\begin{table}[H]
\centering
\begin{tabular}{|p{5cm}|p{9cm}|}
\hline
\textbf{FileName} & \textbf{Purpose} \\ \hline
src/App.jsx & Main component that manages state and renders child components \\ \hline
src/components/Factorial.jsx & Component for factorial calculation with input and display \\ \hline
src/components/Fibonacci.jsx & Component for generating Fibonacci series \\ \hline
src/components/PrimeChecker.jsx & Component for checking if a number is prime \\ \hline
src/App.css & Modern styling with tabs/sections for different mathematical operations \\ \hline
src/main.jsx & Entry point of the React application \\ \hline
vite.config.js & Vite configuration file for build and development \\ \hline
package.json & Project dependencies and scripts configuration \\ \hline
\end{tabular}
\end{table}

\subsection{CONCEPTS USED IN THE APPLICATION}

\begin{table}[H]
\centering
\begin{tabular}{|p{4cm}|p{5cm}|p{4cm}|}
\hline
\textbf{Concept Name} & \textbf{General Purpose} & \textbf{Code File} \\ \hline
Component Composition & Breaking application into reusable components & App.jsx \\ \hline
React Hooks (useState) & Managing state for input numbers and calculation results & Components \\ \hline
Props Passing & Sharing data between parent and child components & All Components \\ \hline
Recursive Functions & Calculating factorial using recursion & Factorial.jsx \\ \hline
Loop Iterations & Generating Fibonacci series & Fibonacci.jsx \\ \hline
Prime Number Algorithm & Checking divisibility for prime detection & PrimeChecker.jsx \\ \hline
Conditional Rendering & Displaying results based on operation selected & All Components \\ \hline
CSS Grid/Flexbox & Multi-section responsive layout & App.css \\ \hline
\end{tabular}
\end{table}

\subsection{ALGORITHM}

\subsubsection{Factorial Calculation}
\begin{enumerate}
    \item Accept a number N from user
    \item Validate N is a non-negative integer
    \item Initialize result = 1
    \item For i from 1 to N: result = result × i
    \item Display the factorial value
    \item Alternative: Use recursive approach - factorial(n) = n × factorial(n-1)
\end{enumerate}

\subsubsection{Fibonacci Series}
\begin{enumerate}
    \item Accept number of terms N from user
    \item Initialize first two terms: a = 0, b = 1
    \item Display first term (0) and second term (1)
    \item For i from 3 to N:
    \begin{itemize}
        \item Calculate next term: c = a + b
        \item Display c
        \item Update: a = b, b = c
    \end{itemize}
\end{enumerate}

\subsubsection{Prime Number Check}
\begin{enumerate}
    \item Accept a number N from user
    \item If N ≤ 1, return "Not Prime"
    \item If N = 2, return "Prime"
    \item For i from 2 to √N:
    \begin{itemize}
        \item If N \% i = 0, return "Not Prime"
    \end{itemize}
    \item Return "Prime"
\end{enumerate}

\subsection{CODE EXPLANATION}
The application implements three mathematical operations in a single React component. For factorial calculation, both iterative and recursive methods are used. The Fibonacci series is generated using a loop that maintains the last two numbers and calculates subsequent terms. Prime number checking uses trial division up to the square root of the number for efficiency. The UI features tabs or sections to switch between different operations, with input validation to ensure valid numerical input.

\subsection{OUTPUT}
\begin{figure}[H]
    \centering
    \fbox{\includegraphics[width=0.8\textwidth]{ass_01.png}}
    \caption{Mathematical Operations Application - Factorial, Fibonacci, Prime}
\end{figure}

\textbf{Test Cases:}
\begin{itemize}
    \item \textbf{Factorial:} Input: 5 → Output: 120 (5! = 5×4×3×2×1)
    \item \textbf{Fibonacci:} Input: 7 → Output: 0, 1, 1, 2, 3, 5, 8
    \item \textbf{Prime:} Input: 17 → Output: Prime Number
    \item \textbf{Prime:} Input: 18 → Output: Not a Prime Number
\end{itemize}

\newpage

% ============ EXERCISE 2 ============
\section{EXERCISE NO: 2}
\label{ex2}

\subsection{AIM}
To create a ReactJS application that reads a number from the user and calculates the sum of its digits.

\subsection{GitHub Repository}
\href{https://github.com/9059Rohith/full_stack_assignment/tree/main/ass_02/ass_02}{https://github.com/9059Rohith/full\_stack\_assignment/tree/main/ass\_02/ass\_02}

\subsection{Live Deployment}
\href{https://melodious-crostata-e74b46.netlify.app/}{https://melodious-crostata-e74b46.netlify.app/}

\subsection{LIST OF FILE NAMES WITH ITS PURPOSE}

\begin{table}[H]
\centering
\begin{tabular}{|p{5cm}|p{9cm}|}
\hline
\textbf{FileName} & \textbf{Purpose} \\ \hline
src/App.jsx & Main component managing calculator state and logic \\ \hline
src/components/InputSection.jsx & Component for number input with validation \\ \hline
src/components/ResultDisplay.jsx & Component to display sum and digit breakdown \\ \hline
src/App.css & Styling for the sum of digits calculator with modern design \\ \hline
src/main.jsx & React application entry point with root rendering \\ \hline
vite.config.js & Vite configuration for development and production \\ \hline
package.json & Dependencies including React, Vite, and development tools \\ \hline
\end{tabular}
\end{table}

\subsection{CONCEPTS USED IN THE APPLICATION}

\begin{table}[H]
\centering
\begin{tabular}{|p{4cm}|p{5cm}|p{4cm}|}
\hline
\textbf{Concept Name} & \textbf{General Purpose} & \textbf{Code File} \\ \hline
React useState Hook & Managing input number and sum result state & App.jsx \\ \hline
String Manipulation & Converting number to individual digits & App.jsx \\ \hline
Array Methods & Using split(), map(), and reduce() for calculation & App.jsx \\ \hline
Conditional Rendering & Showing result based on calculation & App.jsx \\ \hline
Event Handlers & Handling button clicks and input changes & App.jsx \\ \hline
CSS Animations & Smooth transitions and hover effects & App.css \\ \hline
\end{tabular}
\end{table}

\subsection{ALGORITHM}
\begin{enumerate}
    \item Accept a number from the user
    \item Convert the number to a string to access individual digits
    \item Split the string into an array of digit characters
    \item Initialize sum variable to 0
    \item For each digit character:
    \begin{itemize}
        \item Convert the character to an integer
        \item Add the integer value to sum
    \end{itemize}
    \item Display the sum of all digits
    \item Show the breakdown of individual digits
\end{enumerate}

\subsection{CODE EXPLANATION}
The application takes a number as input and calculates the sum of all its digits. It converts the number to a string using toString(), then splits it into an array of individual digit characters. Using the reduce() method or a simple loop, each digit is converted to a number and added to the running sum. The result displays both the individual digits and their total sum. For example, for input 12345, the output would show: 1 + 2 + 3 + 4 + 5 = 15.

\subsection{OUTPUT}
\begin{figure}[H]
    \centering
    \fbox{\includegraphics[width=0.8\textwidth]{ass_02.png}}
    \caption{Sum of Digits Calculator Application}
\end{figure}

\textbf{Test Cases:}
\begin{itemize}
    \item Input: 12345 → Output: Sum = 15 (1 + 2 + 3 + 4 + 5)
    \item Input: 999 → Output: Sum = 27 (9 + 9 + 9)
    \item Input: 1024 → Output: Sum = 7 (1 + 0 + 2 + 4)
    \item Input: 87654 → Output: Sum = 30 (8 + 7 + 6 + 5 + 4)
\end{itemize}

\newpage

% ============ EXERCISE 3 ============
\section{EXERCISE NO: 3}
\label{ex3}

\subsection{AIM}
Design an Employee Tax Calculator using ReactJS to calculate salary components (DA, HRA, Special Allowance), determine employee grade based on total salary, and compute bonus based on the grade. Implement the application using both Class Component and Function Component approaches (Set-1 for odd roll numbers, Set-2 for even roll numbers).

\subsection{GitHub Repository}
\href{https://github.com/9059Rohith/full_stack_assignment/tree/main/ass_03}{https://github.com/9059Rohith/full\_stack\_assignment/tree/main/ass\_03}

\subsection{Live Deployment}
\href{https://dashing-sfogliatella-6fc394.netlify.app/}{https://dashing-sfogliatella-6fc394.netlify.app/}

\subsection{LIST OF FILE NAMES WITH ITS PURPOSE}

\begin{table}[H]
\centering
\begin{tabular}{|p{5cm}|p{9cm}|}
\hline
\textbf{FileName} & \textbf{Purpose} \\ \hline
src/App.jsx & Main component with salary calculator (Function Component approach) \\ \hline
src/ClassComponent.jsx & Alternative implementation using Class Component (if applicable) \\ \hline
src/App.css & Comprehensive styling with gradient backgrounds and modern UI \\ \hline
src/main.jsx & Application entry point with React 18 features \\ \hline
src/index.css & Global styles and Tailwind CSS imports \\ \hline
package.json & Dependencies including React, Lucide icons, and Tailwind CSS \\ \hline
tailwind.config.js & Tailwind CSS configuration for custom styling \\ \hline
vite.config.js & Vite configuration with React plugin \\ \hline
\end{tabular}
\end{table}

\subsection{CONCEPTS USED IN THE APPLICATION}

\begin{table}[H]
\centering
\begin{tabular}{|p{4cm}|p{5cm}|p{4cm}|}
\hline
\textbf{Concept Name} & \textbf{General Purpose} & \textbf{Code File} \\ \hline
React State Management & Managing basic pay, grade, and bonus states & App.jsx \\ \hline
Computed Properties & Auto-calculating DA, HRA, Special Allowance & App.jsx \\ \hline
Conditional Rendering & Showing grade and bonus sections dynamically & App.jsx \\ \hline
Event Handlers & Check Grade and Check Bonus button functions & App.jsx \\ \hline
Lucide React Icons & Professional icons for UI enhancement & App.jsx \\ \hline
Tailwind CSS & Utility-first CSS for rapid UI development & App.jsx, index.css \\ \hline
Input Validation & Ensuring positive numbers only & App.jsx \\ \hline
\end{tabular}
\end{table}

\subsection{ALGORITHM}

\subsubsection{Salary Calculation}
\begin{enumerate}
    \item Accept Basic Pay from user
    \item Calculate DA = Basic Pay × 30\%
    \item Calculate HRA = Basic Pay × 10\%
    \item Calculate Special Allowance = Basic Pay × 5\%
    \item Total Salary = Basic Pay + DA + HRA + Special Allowance
    \item Display all components in breakdown format
\end{enumerate}

\subsubsection{Grade Determination}
\begin{enumerate}
    \item Check Total Salary range:
    \begin{itemize}
        \item If 10,000 - 20,000: Grade = A
        \item If 20,001 - 30,000: Grade = B
        \item If 30,001 - 40,000: Grade = C
        \item If > 40,000: Grade = EXC
    \end{itemize}
    \item Display the grade with visual styling
\end{enumerate}

\subsubsection{Bonus Calculation}
\begin{enumerate}
    \item Check employee grade:
    \begin{itemize}
        \item Grade A: Bonus = Basic Pay × 15\%
        \item Grade B: Bonus = Basic Pay × 12\%
        \item Grade C: Bonus = Basic Pay × 6\%
        \item Grade EXC: Bonus = Basic Pay × 5\%
    \end{itemize}
    \item Display bonus amount with percentage information
\end{enumerate}

\subsection{CODE EXPLANATION}
The application is implemented using Function Components with React Hooks (useState for state management). The salary components are calculated using a computed function that runs on every render when basicPay changes. The grade is determined by the checkGrade function which evaluates the total salary against predefined ranges. The bonus is calculated based on the grade using a switch statement that applies different percentage rates.

\textbf{Function Component Approach:} Uses useState hook for managing state (basicPay, grade, bonus) and functional programming patterns for calculations.

\textbf{Class Component Approach (Alternative):} Would use this.state and this.setState() for state management, with lifecycle methods like componentDidMount() if needed.

The UI is built with Tailwind CSS for responsive design and includes professional icons from Lucide React. Input validation prevents negative numbers and non-numeric inputs. The application features smooth animations and gradient backgrounds for an enhanced user experience.

\subsection{SALARY BREAKDOWN FORMULA}
\begin{itemize}
    \item \textbf{DA (Dearness Allowance)} = Basic Pay × 0.30
    \item \textbf{HRA (House Rent Allowance)} = Basic Pay × 0.10
    \item \textbf{Special Allowance} = Basic Pay × 0.05
    \item \textbf{Total Salary} = Basic Pay + DA + HRA + Special Allowance
    \item \textbf{Total Salary} = Basic Pay × (1 + 0.30 + 0.10 + 0.05) = Basic Pay × 1.45
\end{itemize}

\subsection{OUTPUT}
\begin{figure}[H]
    \centering
    \fbox{\includegraphics[width=0.9\textwidth]{ass_03.png}}
    \caption{Employee Tax Calculator - Main Interface}
\end{figure}

\textbf{Example Calculation:}
\begin{itemize}
    \item Basic Pay: ₹25,000
    \item DA (30\%): ₹7,500
    \item HRA (10\%): ₹2,500
    \item Special Allowance (5\%): ₹1,250
    \item Total Salary: ₹36,250
    \item Grade: C (30,001 - 40,000 range)
    \item Bonus (6\%): ₹1,500
\end{itemize}

\newpage

% ============ EXERCISE 4 ============
\section{EXERCISE NO: 4}
\label{ex4}

\subsection{AIM}
To implement a Dual-Mode Calculator Program using ReactJS that performs:
\begin{itemize}
    \item Basic arithmetic operations (Addition, Subtraction, Multiplication, Division)
    \item String manipulation operations (minimum 8 functions)
    \item Mode toggle between Calculator and String Tools
\end{itemize}

\subsection{GitHub Repository}
\href{https://github.com/9059Rohith/full_stack_assignment/tree/main/ass_04/ass_04}{https://github.com/9059Rohith/full\_stack\_assignment/tree/main/ass\_04/ass\_04}

\subsection{Live Deployment}
\href{https://stellular-parfait-78b8bd.netlify.app/}{https://stellular-parfait-78b8bd.netlify.app/}

\subsection{LIST OF FILE NAMES WITH ITS PURPOSE}

\begin{table}[H]
\centering
\begin{tabular}{|p{5cm}|p{9cm}|}
\hline
\textbf{FileName} & \textbf{Purpose} \\ \hline
src/App.jsx & Main component managing dual-mode state (calculator + string tools), 12 string functions, and arithmetic operations logic \\ \hline
src/components/Display.jsx & Component for displaying mode indicator, input history and current result \\ \hline
src/components/ButtonGrid.jsx & Component containing calculator buttons OR string manipulation buttons based on mode \\ \hline
src/App.css & Modern dual-mode styling with glassmorphism effects, mode toggle button, and string function buttons \\ \hline
src/main.jsx & React application bootstrap and rendering \\ \hline
vite.config.js & Build configuration for Vite bundler \\ \hline
package.json & Project metadata and dependencies \\ \hline
index.html & Root HTML file with app container \\ \hline
\end{tabular}
\end{table}

\subsection{CONCEPTS USED IN THE APPLICATION}

\begin{table}[H]
\centering
\begin{tabular}{|p{4cm}|p{5cm}|p{4cm}|}
\hline
\textbf{Concept Name} & \textbf{General Purpose} & \textbf{Code File} \\ \hline
Component Composition & Separating display and button logic into components & App.jsx \\ \hline
React useState Hook & Managing mode (calc/string), display and operation state & App.jsx \\ \hline
Mode Toggle Pattern & Switching between calculator and string manipulation modes & App.jsx \\ \hline
Props Drilling & Passing multiple functions and data to child components & App.jsx, Components \\ \hline
Event Handlers & Handling button clicks for numbers, operations, and string functions & ButtonGrid.jsx \\ \hline
Arithmetic Operations & Performing +, -, ×, ÷ calculations & App.jsx \\ \hline
String Methods & toUpperCase, toLowerCase, reverse, trim, replace, split & App.jsx \\ \hline
eval() Function & Evaluating mathematical expressions & App.jsx \\ \hline
Conditional Rendering & Displaying different button grids based on mode & ButtonGrid.jsx \\ \hline
CSS Grid Layout & Creating dual layout for calculator and string tools & App.css \\ \hline
Glassmorphism Design & Modern UI with backdrop blur effects & App.css \\ \hline
Error Handling & Managing division by zero and invalid operations & App.jsx \\ \hline
Regular Expressions & Pattern matching for string manipulation (vowels, spaces) & App.jsx \\ \hline
\end{tabular}

\subsubsection{Main Application Flow}
\begin{enumerate}
    \item Initialize state: mode = "calc", input = "", result = "0"
    \item Render mode toggle button (Calculator ↔ String Tools)
    \item If mode = "calc": Show calculator interface
    \item If mode = "string": Show string manipulation interface
    \item Clear input and result when switching modes
\end{enumerate}

\subsubsection{Calculator Mode Algorithm}
\begin{enumerate}
    \item Create calculator UI with number buttons (0-9) and operation buttons (+, -, ×, ÷)
    \item When number button clicked:
    \begin{itemize}
        \item Append digit to display
        \item Update display state
    \end{itemize}
    \item When operation button clicked:
    \begin{itemize}
        \item Store current value and operation
        \item Continue expression building
    \end{itemize}
    \item When equals button clicked:
    \begin{itemize}
        \item Retrieve stored value and operation
        \item Perform calculation using eval()
        \item Display result
    \end{itemize}
    \item Include Clear (C) button to reset calculator
    \item Han

\subsection{STRING MANIPULATION FUNCTIONS (12 Functions)}
dual-mode calculator application is built using a component-based architecture with mode switching capability. The main App.jsx manages state using React hooks (useState) for mode, input, and result values.

\textbf{Component Structure:}
\begin{itemize}
    \item \textbf{App Component:} 
    \begin{itemize}
        \item Manages mode state ("calc" or "string")
        \item Contains 12+ string manipulation functions
        \item Contains arithmetic calculation functions
        \item Handles mode toggle with state reset
    \end{itemize}
    \item \textbf{Display Component:} 
    \begin{itemize}
        \item Receives mode, history, and current values as props
        \item Displays mode indicator (Calculator Mode_calc.png}}
    \caption{Dual-Mode Calculator - Calculator Mode Interface}
\end{figure}

\begin{figure}[H]
    \centering
    \fbox{\includegraphics[width=0.8\textwidth]{ass_04_string.png}}
    \caption{Dual-Mode Calculator - String Manipulation Mode Interface}
\end{figure}

\textbf{Calculator Mode Test Cases:}
\begin{itemize}
    \item Input: 15 + 25 → Output: 40
    \item Input: 100 - 35 → Output: 65
    \item Input: 12 × 8 → Output: 96 (using eval: 12 * 8)
    \item Input: 144 ÷ 12 → Output: 12 (using eval: 144 / 12)
    \item Input: 10 ÷ 0 → Output: Error (Division by zero)
    \item Input: 5 + 3 * 2 → Output: 11 (order of operations)
\end{itemize}

\textbf{String Manipulation Test Cases:}
\begin{itemize}
    \item Input: "hello world" → UPPER → "HELLO WORLD"
    \item Input: "REACT APP" → lower → "react app"
    \item Input: "javascript" → Reverse → "tpircsavaj"
    \item Input: "hello world" → Length → "Length: 11"
    \item Input: "  spaces  " → Trim → "spaces"
    \item Input: "hello" → Capital → "Hello"
    \item Input: "a b c" → No Space → "abc"
    \item Input: "hello world test" → Count → "Words: 3"
    \item Input: "hello world" → Title → "Hello World"
    \item Input: "beautiful" → *Vowels → "b**t*f*l"
    \item Input: "testing" → aLtErNaTe → "TeStInG"

\textbf{Key Implementation Details:}
\begin{itemize}
    \item Mode toggle clears input/result for clean transition
    \item Calculator uses eval() for arithmetic expression evaluation
    \item String functions use native JavaScript methods and regex
    \item Error handling for division by zero and invalid operations
    \item Modern glassmorphism UI with distinct styling for each mode
    \item String function buttons styled differently (purple theme)
    \item Calculator buttons maintain green theme
\end{itemize}
Trim & Remove leading/trailing spaces & "  hi  " → "hi" \\ \hline
Capital & Capitalize first letter & "hello" → "Hello" \\ \hline
No Space & Remove all spaces & "a b c" → "abc" \\ \hline
Count & Count words & "hello world" → "Words: 2" \\ \hline
Title & Title case each word & "hello world" → "Hello World" \\ \hline
*Vowels & Replace vowels with * & "hello" → "h*ll*" \\ \hline
aLtErNaTe & Alternate case pattern & "hello" → "HeLlO" \\ \hline
Space & Add space character & Adds " " to input \\ \hline
\end{tabular}
\end{table}dle errors (division by zero, invalid expressions)
    \item Support decimal numbers and chaining operations
\end{enumerate}

\subsubsection{String Manipulation Mode Algorithm}
\begin{enumerate}
    \item Display text input area for user to type strings
    \item Create grid of 12+ string manipulation buttons
    \item When string function button clicked:
    \begin{itemize}
        \item Apply corresponding string operation to input
        \item Update result display with transformed string
    \end{itemize}
    \item String Functions:
    \begin{itemize}
        \item UPPER: Convert to uppercase using toUpperCase()
        \item lower: Convert to lowercase using toLowerCase()
        \item Reverse: Reverse string using split().reverse().join()
        \item Length: Display character count using .length
        \item Trim: Remove spaces using trim()
        \item Capital: Capitalize first letter
        \item No Space: Remove all spaces using replace(/\\s+/g, '')
        \item Count: Count words using split and filter
        \item Title: Title case each word
        \item *Vowels: Replace vowels with asterisks using regex
        \item aLtErNaTe: Alternate uppercase/lowercase
        \item Space: Add space character
    \end{itemize}
    \item Maintain input and allow chaining operation
    \item Include Clear (C) button to reset calculator
    \item Handle errors (division by zero, invalid expressions)
    \item Support decimal numbers and negative values
\end{enumerate}

\subsection{CALCULATOR OPERATIONS}
\begin{itemize}
    \item \textbf{Addition:} result = operand1 + operand2
    \item \textbf{Subtraction:} result = operand1 - operand2
    \item \textbf{Multiplication:} result = operand1 × operand2
    \item \textbf{Division:} result = operand1 ÷ operand2 (check for zero)
\end{itemize}

\subsection{CODE EXPLANATION}
The calculator application is built using a component-based architecture. The main App.jsx manages state using React hooks (useState) for input and result values. It defines handler functions (updateCalc, calculate, clear, deleteLast) that are passed as props to child components.

\textbf{Component Structure:}
\begin{itemize}
    \item \textbf{Display Component:} Receives history and current values as props, displays them in a clean format
    \item \textbf{ButtonGrid Component:} Receives all handler functions as props, renders calculator buttons in a grid layout
    \item \textbf{App Component:} Manages state, contains calculation logic, and orchestrates child components
\end{itemize}

The calculator uses eval() for expression evaluation, includes error handling for division by zero, and features a modern glassmorphism UI design with smooth animations.

\subsection{OUTPUT}
\begin{figure}[H]
    \centering
    \fbox{\includegraphics[width=0.8\textwidth]{ass_04.png}}
    \caption{Calculator Program Interface}
\end{figure}

\textbf{Test Cases:}
\begin{itemize}
    \item Input: 15 + 25 → Output: 40
    \item Input: 100 - 35 → Output: 65
    \item Input: 12 × 8 → Output: 96
    \item Input: 144 ÷ 12 → Output: 12
    \item Input: 10 ÷ 0 → Output: Error (Division by zero)
\end{itemize}

\newpage

% ============ EXERCISE 5 ============
\section{EXERCISE NO: 5}
\label{ex5}

\subsection{AIM}
Implement a Calculator Program with Game Concept for Kids using ReactJS. Create an interactive educational game where children solve math problems through a fun gaming interface, combining calculator functionality with engaging gameplay elements.

\subsection{GitHub Repository}
\href{https://github.com/9059Rohith/full_stack_assignment/tree/main/ass_05/ass_05}{https://github.com/9059Rohith/full\_stack\_assignment/tree/main/ass\_05/ass\_05}

\subsection{Live Deployment}
\href{https://chic-palmier-4d509b.netlify.app/}{https://chic-palmier-4d509b.netlify.app/}

\subsection{LIST OF FILE NAMES WITH ITS PURPOSE}

\begin{table}[H]
\centering
\begin{tabular}{|p{5cm}|p{9cm}|}
\hline
\textbf{FileName} & \textbf{Purpose} \\ \hline
src/App.jsx & Main game component managing game state and orchestrating gameplay \\ \hline
src/components/GameMenu.jsx & Start screen component with game introduction \\ \hline
src/components/GamePlay.jsx & Active gameplay component with math problems \\ \hline
src/components/GameOver.jsx & End screen showing final score and replay option \\ \hline
src/components/Asteroid.jsx & Individual asteroid component with math problem \\ \hline
src/App.css & Kid-friendly styling with colorful design, animations, and game elements \\ \hline
src/main.jsx & Application entry point and game initialization \\ \hline
package.json & Dependencies including React, Lucide icons for game UI \\ \hline
vite.config.js & Development server and build configuration \\ \hline
index.html & Root HTML with game container and meta tags \\ \hline
\end{tabular}
\end{table}

\subsection{CONCEPTS USED IN THE APPLICATION}

\begin{table}[H]
\centering
\begin{tabular}{|p{4cm}|p{5cm}|p{4cm}|}
\hline
\textbf{Concept Name} & \textbf{General Purpose} & \textbf{Code File} \\ \hline
React useState Hook & Managing game state, score, lives, level & App.jsx \\ \hline
React useEffect Hook & Game loop, asteroid generation, collision detection & App.jsx \\ \hline
React useRef Hook & Managing game loop and input focus & App.jsx \\ \hline
Math.random() & Generating random math problems & App.jsx \\ \hline
setInterval/clearInterval & Game loop timing and animation & App.jsx \\ \hline
Array Methods & Managing asteroids and particles arrays & App.jsx \\ \hline
Conditional Rendering & Different screens (menu, playing, game over) & App.jsx \\ \hline
CSS Animations & Asteroid movement, explosions, particles & App.css \\ \hline
Event Handling & Keyboard input for answers & App.jsx \\ \hline
Game State Machine & Managing game flow and transitions & App.jsx \\ \hline
\end{tabular}
\end{table}

\subsection{GAME MECHANICS}

\subsubsection{Core Features}
\begin{itemize}
    \item \textbf{Math Problems:} Addition problems with increasing difficulty
    \item \textbf{Lives System:} Player starts with 3 lives
    \item \textbf{Level Progression:} Difficulty increases with each level
    \item \textbf{Combo System:} Consecutive correct answers increase multiplier
    \item \textbf{Particle Effects:} Visual feedback for correct/incorrect answers
    \item \textbf{Score System:} Points awarded based on speed and accuracy
\end{itemize}

\subsubsection{Problem Generation}
\begin{enumerate}
    \item Calculate maximum number based on level: maxNum = min(10 + level × 5, 50)
    \item Generate two random numbers between 1 and maxNum
    \item Create addition problem: num1 + num2
    \item Store correct answer for validation
\end{enumerate}

\subsection{ALGORITHM}

\subsubsection{Game Initialization}
\begin{enumerate}
    \item Set initial game state: menu
    \item Initialize score = 0, level = 1, lives = 3
    \item Prepare first math problem
\end{enumerate}

\subsubsection{Game Loop}
\begin{enumerate}
    \item Generate asteroids at random intervals
    \item Move asteroids across screen
    \item Check for user answer submission
    \item If answer correct:
    \begin{itemize}
        \item Increase score
        \item Increment combo
        \item Destroy asteroid with explosion effect
        \item Generate new problem
    \end{itemize}
    \item If answer incorrect or asteroid reaches end:
    \begin{itemize}
        \item Decrease lives
        \item Reset combo
        \item Check for game over condition
    \end{itemize}
    \item Check for level progression (every 10 points)
    \item Update all game elements
\end{enumerate}

\subsubsection{Collision Detection}
\begin{enumerate}
    \item Track asteroid positions
    \item Check if asteroid reaches rocket position
    \item If collision and wrong answer: lose life
    \item If correct answer before collision: destroy asteroid
\end{enumerate}

\subsection{SCORING SYSTEM}
\begin{itemize}
    \item \textbf{Base Score:} 10 points per correct answer
    \item \textbf{Combo Multiplier:} Additional points for consecutive answers
    \item \textbf{Level Bonus:} Harder problems give more points
    \item \textbf{Speed Bonus:} Faster answers earn extra points
\end{itemize}

\subsection{CODE EXPLANATION}
The Space Math Adventure game is a complex React application that combines educational content with engaging gameplay. It uses multiple React hooks for state management:

\begin{itemize}
    \item \textbf{useState:} Manages game state (menu/playing/gameOver), score, lives, level, current problem, user answer, asteroids array, combo counter
    \item \textbf{useEffect:} Implements the game loop with setInterval, handles asteroid spawning, movement, and collision detection
    \item \textbf{useRef:} Maintains reference to the game loop interval and input field for automatic focus
\end{itemize}

The game generates math problems dynamically based on the current level, with difficulty scaling. Asteroids are rendered as objects with position, speed, and problem data. The particle system creates visual explosions using CSS animations when asteroids are destroyed. The application demonstrates advanced React patterns including game loop implementation, dynamic rendering, and complex state management.

\subsection{TECHNICAL FEATURES}
\begin{itemize}
    \item Responsive design for various screen sizes
    \item Smooth 60 FPS animations using CSS transforms
    \item Keyboard input handling for answer submission
    \item Dynamic difficulty adjustment
    \item Particle system for visual effects
    \item Sound-ready architecture (can add audio)
    \item High score tracking capability
    \item Pause/Resume functionality
\end{itemize}

\subsection{OUTPUT}
\begin{figure}[H]
    \centering
    \fbox{\includegraphics[width=0.9\textwidth]{ass05_a.png}}
    \caption{Space Math Adventure - Main Menu}
\end{figure}

\begin{figure}[H]
    \centering
    \fbox{\includegraphics[width=0.9\textwidth]{ass05_b.png}}
    \caption{Space Math Adventure - Gameplay Screen}
\end{figure}

\begin{figure}[H]
    \centering
    \fbox{\includegraphics[width=0.9\textwidth]{ass05_c.png}}
    \caption{Space Math Adventure - Game Over Screen}
\end{figure}

\textbf{Game Flow Example:}
\begin{enumerate}
    \item Player starts game from menu
    \item Problem appears: "7 + 5 = ?"
    \item Player types "12" and submits
    \item Asteroid explodes, score increases by 10
    \item New problem: "13 + 8 = ?"
    \item Continue until lives run out or player quits
\end{enumerate}

\newpage

\section{Conclusion}

This lab manual covers five comprehensive ReactJS exercises that demonstrate:

\begin{itemize}
    \item \textbf{Component Architecture:} Modular design with reusable React components
    \item \textbf{State Management:} Using React Hooks (useState, useEffect, useRef)
    \item \textbf{Props and Data Flow:} Passing data and functions between components
    \item \textbf{Mathematical Operations:} Factorial, Fibonacci, Prime numbers, digit calculations
    \item \textbf{Business Logic:} Salary calculations, tax computations, bonus determination
    \item \textbf{Interactive Applications:} Calculator with real-time evaluation
    \item \textbf{Game Development:} Educational game with gameplay, animation, collision detection
    \item \textbf{UI/UX Design:} Modern styling with Tailwind CSS and custom designs
    \item \textbf{Best Practices:} Component composition, separation of concerns, code reusability
\end{itemize}

All projects are deployed on Netlify and available on GitHub for review and further development.

\subsection{Key Learning Outcomes}
\begin{enumerate}
    \item Understanding React component architecture and hooks
    \item Implementing complex state management strategies
    \item Creating interactive user interfaces with modern CSS
    \item Handling user input and form validation
    \item Building real-world applications with practical use cases
    \item Deploying React applications to production environments
\end{enumerate}

\subsection{Future Enhancements}
\begin{itemize}
    \item Integration with backend APIs (Express.js)
    \item Database connectivity (MongoDB)
    \item User authentication and authorization
    \item Advanced state management (Redux)
    \item Progressive Web App (PWA) features
    \item Testing implementation (Jest, React Testing Library)
\end{itemize}

\vspace{2cm}

\begin{center}
\textbf{End of Lab Manual}
\end{center}

\end{document}